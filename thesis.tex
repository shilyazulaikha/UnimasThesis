\documentclass{UnimasThesis}

\usepackage{graphicx}

\title{<STRATEGIC PLANNING FOR INTEGRATED RAJANG RIVER BASIN DEVELOPMENT>}
\author{<Shilya Zulaikha Binti Zamani>}
\faculty{<Faculty of Engineering>}
\facultyColour{FFCC00} %% 6-digit RGB hexadecimal code 
\submissionyear{2018}
\degreetype{Masters of Engineering\\(Civil Engineering)}

% If using APA bibliography style
\usepackage[natbibapa]{apacite}
\bibliographystyle{apacite}

\begin{document}

\maketitle

\frontmatter

% Acknowledgements from ack.tex
\begin{ACKNOWLEDGEMENTS}
\input{The completion of this study could not have been possible without Allah The Almighty’s blessings and the opportunities that came along throughout the period of this study. A sincere and warmest regards to the ‘Ohana’ which means family in Japanese. This journey is dedicated to my husband, Ahmad Saddam Zamahari, mother, Marinda Mohidin and my father, Zamani Tan Sri Hamdan which has always believed in me and my strongest support system, Allah bless.
A debt of gratitude is also owed to Prof. Dr. Ir. Frederik Joseph Putuhena for pointing us toward the Water Resources and providing me with the guides for my framework. I would also like to sincerely thank Dr. Charles Bong Hin Joo for his expertise in this field and taking time to read my thesis. Together with the lecturers of UNIMAS for guidance and knowledge sharing, I am honored to be your students. Thank you.
Last but not least, to all my dear friends that has been helping and supporting this effort, do accept my warm thank you. Without each and single one of you none of this would indeed be possible.}
\end{ACKNOWLEDGEMENTS}

% English abstract from abstract-en.tex
\begin{enABSTRACT}
\input{Strategic Planning for Integrated Rajang River Basin Development
Shilya Zulaikha Binti Zamani
ABSTRACT
The supervision of water resources must be constructed on sound institutional guidelines and supported institutional measures. Improvements and inventiveness are desiredin the direction ofprovided that an acceptable as well as an allowing environment for operative and competentexecution of Integrated Water Resources Management (IWRM) that will contribute to the understanding of a justifiableand sustainable national water sector. IWRM formulates and implements a progression of action concerning the administration of water and associated resources to accomplishidealprovision of water resources within intended catchment and river basin, whereas protecting besides restoring the environment.Integrated River Basin Management (IRBM) for Rajang river management from a basin perspective and adopting an integrated approach is highly encouraged to optimize benefits so that the environment which produces the ecosystem services we depend on is also protected, and social and economic benefits can be better optimized in the long term. This study indicates that the IRBM implementation is based on 2nd principle of IWRM Plan of Action founded on the Dublin Principles conferred at the World Summit in Rio de Janeiro. The 2nd principle stated that the water resources development and management should be based on a participatory approach, concerning all related stakeholders. The participatory approach is implemented by applying Logical Framework Approach to analyse the framework in Strategic Planning for Integrated Rajang River Basin Development.}
\end{enABSTRACT}

% Malay Abstract from abstrak-ms.tex
\begin{msABSTRACT}
\input{Perancangan Strategik bagi  Pembangunan Lembahan Sungai Rajang Bersepadu
Shilya Zulaikha Binti Zamani
ABSTRAK
Pengurusan sumber air hendaklah berpandukan polisi institusi yang kukuh dan teratur. Pembaharuan dan inisitatif diperlukan ke arah persekitaran yang mencukupi serta persekitaran yang sesuai untuk pelaksanaan Pengurusan Sumber Air Bersepadu (IWRM ke arah sektor air negara yang mampan. IWRM merumuskan dan melaksanakan suatu tindakan yang melibatkan pengurusan sumber air yang berkaitan bagi mencapai peruntukan optimum sumber air tadahan atau lembah sungai, di samping memulihara dan memelihara alam sekitar. Pengurusan Lembahan Sungai Bersepadu (IRBM) bagi Sungai Rajang dengan pendekatan bersepadu adalah sangat digalakkan bagi mengoptimumkan manfaat dan impak alam sekitar. Pengurusan lembahan sungai yang efektif bagi persekitaran di mana kitaran ekosistem yang bergantung kepada kehidupan seharian juga akan dilindungi, serta faedah ekonomi dan sosial juga dapat dioptimumkan dalam jangka masa panjang. Oleh itu, kajian ini berasaskan  pelaksanaan IRBM bagi Sungai Rajang berdasarkan Prinsip ke-2 Pelan Tindakan IWRM yang di asaskan pada ‘Dublin Principles,’ dan dibentangkan di Sidang Kemuncak Dunia di Rio de Janeiro. Prinsip ke-2 menyatakan bahawa pembangunan dan pengurusan sumber air harus berdasarkan pendekatan secara penyertaan, di mana melibatkan semua pihak berkepentingan dan berkaitan. Pendekatan ini yang dilaksanakan akan menggunakan ‘Logical Framework Approach’ bagi menganalisis rangka kerja dalam Perancangan Strategik bagi Pembangunan Lembahan Sungai Rajang Bersepadu.}
\end{msABSTRACT}

\tableofcontents

\listoffigures
\listoftables

% List of Symbols may be prepared as in symbols.tex
\input{symbols}


\mainmatter
% Each chapter from a separate file
\input{chap-intro}
\input{chap-fibonacci}
\input{chap-goldenratio}



% references are listed in refs.bib
\bibliography{refs}

\appendix
% Each appendix chapter from a separate file
\input{app-details}
\input{app-code}
\end{document}
